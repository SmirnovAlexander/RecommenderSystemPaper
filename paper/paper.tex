\documentclass{article}

\usepackage{fontspec}                % font for emoji
\newfontfamily\DejaSans{DejaVu Sans}

\usepackage{arxiv}

\usepackage[utf8]{inputenc} % allow utf-8 input
\usepackage[T1]{fontenc}    % use 8-bit T1 fonts
\usepackage{hyperref}       % hyperlinks
\usepackage{url}            % simple URL typesetting
\usepackage{booktabs}       % professional-quality tables
\usepackage{amsfonts}       % blackboard math symbols
\usepackage{nicefrac}       % compact symbols for 1/2, etc.
\usepackage{microtype}      % microtypography
\usepackage{lipsum}		    % Can be removed after putting your text content
\usepackage{graphicx}
\usepackage{natbib}
\usepackage{doi}            % hyperlink to doi.org
\usepackage{enumitem}       % itemize settings
\usepackage{cleveref}       % reference section with capital letter
\usepackage{xcolor}         % additional colors for hypersetup
\usepackage{booktabs}       % beauful tables
\usepackage{caption}        % table caption margin
\usepackage{placeins}       % floatbarrier

\captionsetup[table]{skip=10pt}          % table caption margin
\setlist[itemize]{noitemsep, topsep=0pt} % no space between itemize
\hypersetup{
    colorlinks=true,
    citecolor=violet,
    linkcolor=violet,
    filecolor=violet,      
    urlcolor=black,
}

\title{A hybrid approach for news recommender system using optimization methods}

%\date{September 9, 1985}	% Here you can change the date presented in the paper title
%\date{} 					% Or removing it

\author{
    {\includegraphics[scale=0.07]{./images/spbu.png}\hspace{1mm}Alexander Smirnov}\\
	Department of Information and Analytical Systems\\
	Saint Petersburg State University\\
	Russia, Saint Petersburg\\
	\texttt{ru.alexander.smirnov@gmail.com} \\
	\And
	{\includegraphics[scale=0.07]{images/spbu.png}\hspace{1mm}Elena Mikhailova} \\
	Department of Information and Analytical Systems\\
	Saint Petersburg State University\\
	Russia, Saint Petersburg\\
	\texttt{e.mikhaylova@spbu.ru} \\
	%% \AND
	%% Coauthor \\
	%% Affiliation \\
	%% Address \\
	%% \texttt{email} \\
	%% \And
	%% Coauthor \\
	%% Affiliation \\
	%% Address \\
	%% \texttt{email} \\
	%% \And
	%% Coauthor \\
	%% Affiliation \\
	%% Address \\
	%% \texttt{email} \\
}

% Uncomment to remove the date
%\date{}

% Uncomment to override  the `A preprint' in the header
%\renewcommand{\headeright}{Technical Report}
%\renewcommand{\undertitle}{Technical Report}
\renewcommand{\shorttitle}{Recommender System}

\hypersetup{
    pdftitle={A hybrid approach for news recommender system using optimization methods},
    pdfsubject={recommender systems},
    pdfauthor={Alexander Smirnov, Elena Mikhailova},
    pdfkeywords={Recommender Systems, Optimizations},
}

\begin{document}

    \maketitle

    \begin{abstract}


        Recommender system is an essential part of any social media application. Most recommender systems now use a hybrid approach, combining collaborative filtering, content-based filtering, and other approaches. Most common problems in the field of hybrid recommenders are cold start and data sparsity \citep{overview}. In this paper we address the abovementioned problems by proposing a hybrid weighted news recommender system which combines different approaches.

    \end{abstract}


    \keywords{hybrid recommender systems \and content-based recommender \and collaborative recommender \and optimizations}


    \section{Introduction}

        Recommender system is a crucial part of every application that operates with content and user activity. Enormous amount of information leads to the problem that user is not able to find relevant content.

        Common approaches, such as collaborative filtering, has its own problems: cold start, scalability and data sparsity. Content-based approaches suffer from the fact that we have to somehow represent recommended item in feature space.
        
        To be consistent during the paper we list some domain specific vocabulary with their meanings:

            \begin{itemize}
                \item Rating: expression or preference

                    \begin{itemize}
                        \item explicit (direct from user, e.g. user rated film)
                        \item implicit (inferred from user activity, e.g. user stopped watching movie after 5 minutes)
                    \end{itemize}
                \item Prediction: estimate of preference
                \item Recommendation: selected items for user
                \item Content: attributes, text, etc; everything about item
            \end{itemize}


        The remainder of this paper is organized as follows:
        
            \begin{itemize}
                \item \Cref{sec:related} describes the relevant related work
                \item \Cref{sec:input} describes input data
                \item \Cref{sec:overview} explains our modular design and architecture
                \item \Cref{sec:implementation} describes the implementation of the algorithms in a real system
                \item \Cref{sec:evaluation} provides tests and experiments validating
our system’s results
                \item \Cref{sec:further} explains future work
                \item \Cref{sec:summary} presents conclusions
            \end{itemize}

    \section{Related work}
    \label{sec:related}

    \section{Input data}
    \label{sec:input}

    As we solving domain specific task, we have domain specific data.

    \subsection*{metadata}

    \begin{table}[h]
        \centering
        % \resizebox{\textwidth}{!}{
        \begin{tabular}{ccccc}
            \toprule

            \emph{item\_id} & \emph{date} & \emph{source\_id} \\\midrule

            1 & 2021-01-08 22:08:39 & 9  \\
            2 & 2021-01-09 10:28:58 & 5  \\
            3 & 2021-01-09 14:20:34 & 12 \\
            \vdots & \vdots & \vdots  \\\bottomrule


         \hline
        \end{tabular}
        % }

        \caption{news metadata}
        \label{tab:meta}
    \end{table}

    
    \begin{itemize}
        \item \emph{source\_id}: source of news item
    \end{itemize}


    \subsection*{content}

    \begin{table}[h]
        \centering
        \resizebox{\textwidth}{!}{
        \begin{tabular}{cll}    \toprule

            \emph{item\_id} &  \emph{news title} & \emph{news content} \\\midrule

        1  &  Azerbaijan denies reports on construction of Turkish air bases in the country & Information that Turkey will create air bases ...        \\
        2  &  Durov announced the massive transition of WhatsApp users to Telegram          & Telegram developer Pavel Durov said in his channel ...   \\
        3  &  Passenger plane that disappeared from radar crashed                           & Passenger plane taking off from Jakarta, disappeared ... \\
        \vdots & \vdots & \vdots \\\bottomrule


         \hline
        \end{tabular}
        }

        \caption{news item}
        \label{tab:item}
    \end{table}

    \subsection*{shows \& views}

    \begin{table}[h]
        \parbox{.45\textwidth}{
            \centering
            \begin{tabular}{ccc}
                \toprule

                \emph{user\_id} & \emph{item\_id} \\\midrule
                10 & 1  \\
                10 & 2  \\
                23 & 1  \\
                23 & 3  \\
                23 & 2  \\
                38 & 3  \\
                38 & 1  \\
                \vdots & \vdots  \\\bottomrule
            \end{tabular}
            \caption{news shows}
            \label{tab:show}
        }
        \hfill
        \parbox{.45\textwidth}{
            \centering
            \begin{tabular}{ccc}
                \toprule

                \emph{user\_id} & \emph{item\_id} \\\midrule
                10 & 1  \\
                10 & 2  \\
                23 & 1  \\
                38 & 3  \\
                \vdots & \vdots  \\\bottomrule
            \end{tabular}
            \caption{news views}
            \label{tab:view}
            }
    \end{table}   


    \begin{itemize}
        \item \emph{shows}: if \emph{item\_id} was shown to the \emph{user\_id}
        \item \emph{views}: if \emph{item\_id} was clicked by the \emph{user\_id}
    \end{itemize}

    \newpage
    

    \subsection*{emotions \& comments}

    There is an option to react on item via leaving emotion and/or writing a comment.

    \begin{table}[h]
        \parbox{.45\textwidth}{
            \centering
            \begin{tabular}{ccc}
                \toprule

                \emph{user\_id} & \emph{item\_id} & \emph{emotion\_id} \\\midrule
                10 & 1 & 1  \\
                10 & 2 & 3        \\
                23 & 1 & 3        \\
                38 & 3 & 2        \\
                \vdots & \vdots & \vdots  \\\bottomrule

            \end{tabular}
            \caption{users' emotions}
            \label{tab:emotion}
        }
        \hfill
        \parbox{.45\textwidth}{
            \centering
            \resizebox{0.45\textwidth}{!}{
            \begin{tabular}{ccc}
                \toprule

                \emph{user\_id} & \emph{item\_id} & \emph{comment} \\\midrule
                10 & 1 & that's great  \\
                10 & 1 & wish it will continue        \\
                23 & 2 & whatsapp is not competetive anymore        \\
                \vdots & \vdots & \vdots  \\\bottomrule

            \end{tabular}
        }
            \caption{users' comments}
            \label{tab:comment}
            }
    \end{table}   
    
    \begin{itemize}
        \item \emph{emotion\_id}: one of \{{\DejaSans 😄, 😲, 😞, 😡, ♥ }\}
    \end{itemize}


    % \FloatBarrier


    \subsection*{users' subscriptions}

    If \emph{user\_id} subscribed to the \emph{source\_id}.

    \begin{table}[h]
        \centering
        \begin{tabular}{cc}
            \toprule

            \emph{user\_id} & \emph{source\_id} \\\midrule

            10 & 9  \\
            23 & 5  \\
            \vdots & \vdots  \\\bottomrule

        \end{tabular}

        \caption{users' subscriptions}
        \label{tab:subscriptions}
    \end{table}




    \section{Overview of our approach}
    \label{sec:overview}

    \subsection{Collaborative filtering recommendation}

    For collaborative filtering recommendation we should have something known as user-item matrix which may be formed from user activity from tables (cite tables)

    \subsection{Popularity-based recommendation}

    For measuring news item popularity following data can be aggregated: \hyperref[tab:show]{shows}, \hyperref[tab:view]{views}, \hyperref[tab:emotion]{emotions}, \hyperref[tab:comment]{comments}.

    \begin{table}[h]
        \centering
        % \resizebox{\textwidth}{!}{
        \begin{tabular}{ccccc}
            \toprule

            \emph{item\_id} & \emph{shows\_num} & \emph{views\_num} & \emph{emotions\_num} & \emph{comments\_num} \\\midrule

            1 &  1043 & 231 & 52 & 7  \\
            2 &  828 & 478 & 78 & 11 \\
            3 &  163 & 25 & 5 & 0    \\
            \vdots & \vdots & \vdots & \vdots & \vdots \\\bottomrule


         \hline
        \end{tabular}
        % }

        \caption{aggregated popularity data}
        \label{tab:popularity}
    \end{table}



    \section{Implementation}
    \label{sec:implementation}

    \section{Evaluation}
    \label{sec:evaluation}

    \section{Further research}
    \label{sec:further}

    \section{Summary}
    \label{sec:summary}

        Results show that combining different approaches leads to rise of users' involvement.

        

    \bibliographystyle{unsrtnat}
    \bibliography{references}  

\end{document}
