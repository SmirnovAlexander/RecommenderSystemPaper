\documentclass{article}

\usepackage{arxiv}

\usepackage[utf8]{inputenc} % allow utf-8 input
\usepackage[T1]{fontenc}    % use 8-bit T1 fonts
\usepackage{hyperref}       % hyperlinks
\usepackage{url}            % simple URL typesetting
\usepackage{booktabs}       % professional-quality tables
\usepackage{amsfonts}       % blackboard math symbols
\usepackage{nicefrac}       % compact symbols for 1/2, etc.
\usepackage{microtype}      % microtypography
\usepackage{lipsum}		    % Can be removed after putting your text content
\usepackage{graphicx}
\usepackage{natbib}
\usepackage{doi}



\title{A hybrid approach for news recommender system using optimization methods}

%\date{September 9, 1985}	% Here you can change the date presented in the paper title
%\date{} 					% Or removing it

\author{
    {\includegraphics[scale=0.07]{./images/spbu.png}\hspace{1mm}Alexander Smirnov}\\
	Department of Information and Analytical Systems\\
	Saint Petersburg State University\\
	Russia, Saint Petersburg\\
	\texttt{ru.alexander.smirnov@gmail.com} \\
	\And
	{\includegraphics[scale=0.07]{images/spbu.png}\hspace{1mm}Elena Mikhailova} \\
	Department of Information and Analytical Systems\\
	Saint Petersburg State University\\
	Russia, Saint Petersburg\\
	\texttt{e.mikhaylova@spbu.ru} \\
	%% \AND
	%% Coauthor \\
	%% Affiliation \\
	%% Address \\
	%% \texttt{email} \\
	%% \And
	%% Coauthor \\
	%% Affiliation \\
	%% Address \\
	%% \texttt{email} \\
	%% \And
	%% Coauthor \\
	%% Affiliation \\
	%% Address \\
	%% \texttt{email} \\
}

% Uncomment to remove the date
%\date{}

% Uncomment to override  the `A preprint' in the header
%\renewcommand{\headeright}{Technical Report}
%\renewcommand{\undertitle}{Technical Report}
\renewcommand{\shorttitle}{Recommender System}

\hypersetup{
    pdftitle={A hybrid approach for news recommender system using optimization methods},
    pdfsubject={recommender systems},
    pdfauthor={Alexander Smirnov, Elena Mikhailova},
    pdfkeywords={Recommender Systems, Optimizations},
}

\begin{document}

    \maketitle

    \begin{abstract}

        There are many approaches \citep{approaches} to build recommender system. They have their advantages and disadvantages. In this work we address the question whether it worths to use as many information as we are able to collect to build a hybrid recommender system. This work is aimed to combine multiple recommenders via optimizing their contributions to weights vectors. Results show that combining different approaches leads to rise of users' involvement.

    \end{abstract}


    \keywords{Recommender Systems \and Optimizations}


    \section{Introduction}
    
        Nowadays recommender systems are crucial part of every service that operates with user activity. There is so many information that user needs help to find content he wants to consume.

        In this paper we found that it would be great to combine all available recommenders with different weights to negotiate their weak sides.

        We use \textbf{Collaborative Filtering} and \textbf{Content-Based filtering} as main approaches.

        Previously there were works related to single recommendation method, but there are not many of them that combine all approaches together and optimize them.

        Vocabulary:

            \begin{itemize}
                \item Rating: expression or preference
                \item Explicit (direct from user, e.g. rate the film)
                \item Implicit (inferred from user activity, e.g. stop watching movie after 5 min)
                \item Prediction: estimate of preference
                \item Recommendation: selected items for user
                \item Content: attributes, text, etc; everything about item
                \item Collaborative: using data from other users
            \end{itemize}

        This work aimed to prove that hybrid methods works better than single recommenders. It will fill a gap between content creators and users so it will be easier to deliver content.

        Firstly we will implement all state-of-the-art solutions. Afterwards we will combine them and optimize.



        Results show that combining different approaches leads to rise of users' involvement.


    \bibliographystyle{unsrtnat}
    \bibliography{references}  

\end{document}
